% !TEX root = ../thesis.tex

\chapter{Introduction}
\label{chap:intro}

\section{Outline}
\subsection{Setting}
There are three contexts with which we will be concerned in this thesis: classification, boosting and hedging.
In the classification setting we are expected to produce the correct answer based on some input, while being provided with example problems and their correct answers so we can try to discover the hidden rules. The second setting is boosting which is an answer to the classification problem. Here we start out with a rule of thumb that must perform only slightly better than random guessing. We then proceed to improve the accuracy of this rule of thumb by trying to identify the important examples and adjusting our rule of thumb accordingly. The final context is that of hedging, meaning ``to protect oneself against loss by making balancing or compensating transactions'', hedging is all about cleverly allocating resources to minimize losses. Hedging algorithms are often used as subroutines of boosting algorithms to find the difficult examples of a problem in an attempt to improve accuracy.

\subsection{Scope}

The main goal of this thesis is to compare the practical performance of two previously known algorithms: \adaB, \NHB and one new algorithm called \squintB. These are boosting algorithms based on hedge allocation algorithms called \hedge, \adaN and \squint respectively. In \cite{Koolen2015} Koolen and Van Erven proved that \squint theoretically outperforms both \hedge and \adaN. However, since the theory must account for edge cases which might not occur in the average scenario, it is possible that the difference in practical performance between \squint and the other two algorithms is negligible or, as will turn out to be the case, \squintB might not outperform the previous algorithms at all. In this thesis we will implement all of the above algorithms to compare their accuracy as a function of the number of iterations over two datasets.

\par In this thesis we will only explore the algorithms in a practical setting but since \squint was very recently developed, the curious reader might wonder how \squintB performs theoretically. In \cite{Otten2016} Otten performs a similar exploration of these algorithms but from a theoretical perspective, exploring how \squintB performs theoretically in comparison to the other two. 

\subsection{Basic structure}
In chapter \ref{chap:prelim} we will give a formal description of the settings and procedures we will use throughout the thesis. We will also introduce the notations and basic terminology, as well as briefly discuss the results of previous works, both practical and theoretical. In chapter \ref{chap:pracPerf} we will discuss the results of the algorithms and compare their accuracy.