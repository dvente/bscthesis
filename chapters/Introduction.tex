% !TEX root = ../thesis.tex

\chapter{Introduction}
\label{chap:intro}
\subsection*{Motivation}
\label{sec:motiv}

\paragraph*{}Technology is becoming evermore ubiquitous. In its ubiquity this technology brings many things, but above all it has brought a rise in the supply of data and the value of time and attention, both from a consumer and a business perspective. Here machine learning might provide a solution since, what one might call, ``manual automatization'', i.e. explicitly programming a computer to make certain kinds of decisions, is complex, labour intensive, highly specific and thus expensive and ineffective. Machine learning however teaches quite literally by example. As long as one has the data, by way of machine learning one can teach computers to do things beyond one's own abilities. While better then the ``manual automatization'' general machine learning can still be hard since it depends on criteria and values that might not be obvious. Here we can again provide a solution: boosting, a way to combine several ``good enough'' and usually obvious criteria into a very good one. 

\paragraph*{Scope} In 1997 Freund and Schapire  \cite{Freund1997} published their paper on On-line learning and boosting, introducing \hedge and \ada.
